\documentclass{article}

\usepackage[utf8]{inputenc}
\usepackage[T1]{fontenc}
\renewcommand*\familydefault{\sfdefault}
\usepackage[francais]{babel}
\usepackage{graphicx}
\usepackage{tabularx}
\usepackage{listings}
\usepackage[left=4cm, right=4cm]{geometry}

\title{\textbf{Répartiteur de charge}}
\author{%
  Clément \textsc{Delpech} \\
  Armel   \textsc{Mangean} \\
  Idrissa \textsc{Sokhona}
}%
\date{\today}

\begin{document}

  \maketitle
  \thispagestyle{empty}
  \newpage

  \tableofcontents
  \thispagestyle{empty}
  \newpage
 
  \setcounter{page}{1}

  \section{Analyse}
  \subsection{Définition du sujet}
    
    % Petite intro / Definition dans les très grandes lignes du sujet
    Une problématique récurrente dans le monde informatique est
    l'optimisation des ressources disponibles. Dans un réseau les
    ressources sont souvent inégalement utilisées. Une meilleure
    utilisation de ces ressources pourrait être obtenu en permettant à
    ces machines de s'échanger des tâches. C'est le but d'un système
    de répartition de charge.
        
    Afin de mieux comprendre ce qui est demandé voici quelques
    définitions des termes techniques employés.
    
    \paragraph{Tâche} Dans notre système une tâche l'execution d'un
      programme. On pourra considérer plus tard le degré
      d'interactivité que l'on souhaite gérer (gestion des E/S
      standards, gestion des fichiers, interface graphique...)
      
    %\paragraph{Un n\oe{}ud, un site} Une station de travail UNIX dans
    %  l'implémentation demandé

    \paragraph{Placement réparti} Ordonnancement de programmes i.e. choix
      d'un processeur pour un processus parmis un ensemble de
      machines. C'est une manière de tirer parti des ressources
      inutilisées.
      
    % Dynamique à détailler si on a le temps
    \paragraph{Placement dynamique et statique} Dans le cadre du
      placement statique avant le démarrage du système de répartion sera
      associé à une tâche un processeur. Cette tâche ne pourra
      qu'être executé sur ce processeur même si celui-ci est
      surchargé. Dans le cadre du placement dynamique le processeur où
      s'executera une tâche sera déterminé pendant l'execution.
              
    % TODO bonus - Eventulement accompagner d'un schéma comme celui du cours de Folliot
    \paragraph{Partage et équilibrage de charge} Il existe deux grandes
      stratégies de placement dynamique. La première stratégie est
      nommée partage de charge. Lorsque la charge globale d'une
      machine dépasse un seuil caractérisant la surcharge, il devient
      nécessaire de déplacer quelques unes de ces tâches. Le but du
      partage de charge est de maximiser le débit d'exécution moyen
      des tâches. La seconde stratégie de placement est nommée
      équilibrage de charge. À chaque instant le système de
      répartition essaye de conserver un équilibre dans la répartition
      de la charge globale. Pour cela il determine une charge idéale
      que chacune des machines tend à expérimenter. Cette charge
      idéale est la moyenne des charges individuelles du système.

    %\paragraph{Coopératif} Placement décidé grâce à la coopération
    %  de l'ensemble des machine.

    %\paragraph{Connaissance de l'état global} La coopération permet
    %  d'avoir une réprésentation de l'ensemble du système (cependant
    %  inexacte du au temps de transmissions de données).

    \paragraph{Centralisation} Si l'on considère une implémentation
      décentralisé de ce système alors chaque n\oe{}ud a les mêmes
      prérogatives. Chaqun doit connaitre l'état du système ce qui
      impose soit un échange important de message passant
      difficilement à l'échelle soit un système d'échange
      d'information entre les n\oe{}ud complexe (connaissance
      partielle de l'état du système sur un n\oe{}ud). De plus, cela
      implique une connaissance approximative de l'état du système.
      Une approche centralisé defini les n\oe{}uds comme esclave
      soumis à un n\oe{}ud spécifique coordinateur. Quand un n\oe{}ud
      souhaite transferer une tâche, il envoie une demande au
      coordinateur qui choisit alors un n\oe{}ud cible en utilisant
      l'ensemble des information à sa disposition. On parlera aussi
      d'approche maitre-esclave ou bien client-serveur.      
      
    \paragraph{}
    L'objectif de ce projet est d'aboutir à la mise en \oe{}uvre
    effective d'un ordonnancement de processus sur un ensemble des
    machine. C'est à dire, à un instant donnée, déterminer le
    placement d'une tâche sur la machine la plus capable de la traiter
    en fonction de la charge de chacune des machines du parc. Le
    programme consiste en une couche d'abstraction intermediaire entre
    le système d'exploitation et les programmes à executer.
        
    \begin{figure}[h!]
      \centering
      \includegraphics[width=0.58\textwidth]{img/couches_du_systeme.png}
      \caption{Couches d'abstrations présente dans un noeud}
    \end{figure}
    
  % Dans cette partie on doit définir toutes les problématiques associé
  % càd à quelles questions on devra répondre dans la partie conception
  \subsection{Compréhension des problèmes}
    
    La mise en place d'une telle solution pose un ensemble de
    questions auquelles il est nécessaire de répondre. Il est évoqué
    dans le sujet quatres grandes problématiques lié à la répartition.

    \begin{itemize}
      \item Evaluation de la charge de chaque machine
      \item Definition de la surcharge
      \item Choix de la tâche à déplacer 
      \item Choix des machines
    \end{itemize}

    Pour chacune de ces questions il est nécessaire de définir une
    politique définissant la manière dont on y répond. Aux politique
    répondant à ces questions s'ajoute une politique de tolérance aux fautes.

    \subsubsection{Politique d'information}

      Par définition, le système étant réparti il y a une dispersion
      des données importante. Le module d'information consiste à
      définir les informations devant être collectées, les instants
      pendant lesquels ces informations sont collectées, à partir de
      quels n\oe{}uds elles sont collectées.

      % A completer?
      \paragraph{Évaluation de la charge} Une bonne évaluation de la
        charge est cruciale pour notre appliction, l'efficacité et la
        réactivité de notre application en dépend. La charge de chaque
        machine constitue donc une information essentielle devant être
        recueillie. 

      % Mettre un schema?
      \paragraph{Répartition de l'information} Il existe deux méthodes
        de récolte d'information de la charge. La première est une
        politique où l'information est récoltée à la demande, ce qui
        permet de minimiser le nombre de messages échangés. Le
        principal inconvénient étant que l'information peut ne pas
        être suffisamment à jour dans le cas où les évenements qui
        lancent la récolte ne se produisent pas assez souvent. La
        seconde est une politique où l'information est récoltée
        périodiquement. Avec cette méthode la définition de la période
        est cruciale, trop grande on retrouvera le même inconvenient
        qu'avec une récolte évenementielle, trop petite les messages
        de contrôles risquent de saturer le réseau.
      
    \subsubsection{Politique de transfert}

      Le module de transfert est responsable de déterminer si un
      n\oe{}ud est dans un état approprié pour participer à un
      transfert de tâche comme source ou comme receveur.

      \paragraph{Évalutation de la surcharge} Afin de décider s'il est
        possible ou necessaire de faire un transfert il est possible
        de se référer à un seuil au delà du quel on considère être
        surchargé. Ce seuil peut être fixe ou bien dynamique,
        dépendant de l'activité du système de répartition.

      \paragraph{Décision de placement} Dans le cas du partage de
        charge, lors du dépassement de seuil de surcharge il est
        necessaire d'emetter une décision de placement pour chaque
        nouvelle tâche.  Dans le cas de l'équilibrage de charge un
        décision est requise à chaque nouvelle tâche sans contrainte
        de seuil.

    \subsubsection{Politique de localisation}

      Ce module est responsable de trouver les n\oe{}uds, émetteur et
      recepteur, étant les meilleures candidats au transfert d'une
      tâche. On répond donc ici au problème du choix de la machine
      réceptrice.

      \paragraph{Approche serveur} La première approche basée sur
        un modèle serveur consiste à permettre aux machines pouvant
        accueillir des tâches supplémentaires de proposer leurs
        services. Lorsqu'une machine estime pouvoir accueillir une
        tâche elle émet une offre de service. Les n\oe{}uds subissant
        une surcharge répondent en conséquence. La machine détermine
        alors le n\oe{}ud le plus approprié à ce transfert de tâche.

      \paragraph{Approche client} La seconde approche, à l'inverse,
        se base sur le modèle \textit{client} et consiste à permettre
        aux machines subissant une surcharge de demander les services
        d'une des autres machines du système. Lorsqu'une machine
        souhaite transferer une tâche elle emet une demande de partage
        de charge. Les n\oe{}uds aptes à offrir leur aide répondent en
        conséquence. La machine détermine alors le n\oe{}ud le plus
        approprié.
          
     %\paragraph{} Dans la pratique le choix d'un approche dépendra
     %   fortement si l'on est dans un système avec centralisation ou
            
    %\subsubsection{Politique de séléction}
    %  La politique de sélection est responsable du choix des tâches à
    %  transférer. Nous distinguons trois politiques principales de
    %  sélection.

    %  \begin{itemize}
    %    \item choisir une des tâches ayant contribué à ce que le nœud 
    %            devienne surchargé.
    %    \item choisir n'importe quel tâche c'est à dire toute les 
    %            tâches sont considérées comme éligibles.
    %    \item choisir une tâche approprié. Le choix d'une tâche 
    %           appropriée peut nécessiter une bonne connaissance sur la
    %            tâche aussi bien que sur les machines destinataire.
    %  \end{itemize}
        
    % NB on doit gerer une machine qui tombe, mais pas un remise en route auto
    \subsubsection{Politique de tolérance aux fautes}

      On doit gérer les pannes franches des machines, c'est-à-dire les
      machines ne répondant plus. D'une part on doit pouvoir détecter
      la panne ce qui implique d'interoger chaque machine à intervalle
      de temps réguilier. D'autre part, on doit faire en sorte que la
      panne ne pertube pas le bon déroulement de notre système. On
      pourra donc spécifier explicitement que les tâches interrompues
      ne se sont pas terminée correctement. Ainsi, la machine les
      ayant transféré pourra de nouveau les faire executer dans le
      système réparti.

\newpage
\section{Conception}

  Dans cette partie, nous allons nous intéresser aux données présentes
  sur chaque n\oe{}ud, aux messages envoyés entre sites et aux
  algorithmes de chacun des modules étudiés ci-dessus.

  \subsection{Topologie}

    Par soucis de simplicité et de faisabilité -- n'ayant jamais
    implémenté de système réparti -- nous organiserons notre système
    de placement selon un modèle maître-esclaves. En effet, la
    centralisation des données simplifie la réalisation d'un tel
    système. Le rôle d'un esclave est de participer au placement
    réparti nottamment en exécutant les tâches qui lui sont
    confiées. Il a connaissance de l'identité du maître ainsi que de
    sa propre charge. Le rôle du maître est de coordoner le placement
    réparti. Ainsi, le placement est réalisé en centralisant les
    données ainsi que la prise de décision. Le maître a connaissance
    de tous les esclaves ainsi que de la charge de chacun des
    esclaves. La machine maître endosse le double rôle de maître
    et d'esclave.
        
    \begin{figure}[h!]
      \centering
      \includegraphics[scale=0.45]{img/topologie.pdf}
      \caption{Topologie du système de placement réparti}
    \end{figure}
      
    % petit argumentation sur les autre possibilité et les raison du
    % choix

    Notons que cette solution peut poser problème dans le cas d'une
    machine maitre peu performante ou d'un nombre d'esclave trop
    grand. Un système totalement décentralisé effacerai le problème de
    goulot d'étrangelement au niveau du maîtere. Compte-tenu des
    exigences de notre projet, on peut imaginer que le nombre de
    machines restera raisonnable (inférieure à 20), ainsi la gestion
    des esclaves par un ordinateur de bureau récent restera trés
    largement supportable.
        
  \subsection{Interface utilisateur}

    Concernant les interaction avec l'utilisateur nous avons choisi la
    voie de la simplicité. L'application se lancera via un terminal.
    A l'interieur de l'application l'interface sera en ligne de
    commande avec un nombre de commandes réduit. Chaque commande
    traitera une opération bien distincte.

    \begin{itemize}
      \item S'integer au système
      \item Executer une tache dans système réparti
      \item Quitter l'application
      \item Observer le système
    \end{itemize}
        
  \subsection{Calcul de la charge d'une machine}

    Il est capital de définir correctement ce que sera la charge dans
    notre application afin d'obtenir de bonnes performances. Une
    solution simple est de considérer le nombre de processus à l'état
    prêt, attendent le processeur. Il faudra considérer l'ensemble des
    processus c'est-à-dire aussi ceux s'executant en dehors du système
    de placement réparti. Afin de renforcer la validité d'utilisation
    de ce facteur on pourra le pondérer en fonction des particularités
    de chaque machine.
      
    \subsection{Module d'information}

      Ce module concerne la gestion de l'état ainsi que de la charge
      du système. Compte-tenu de la topologie choisie, les
      informations gobales seront centralisés au niveau du maître. Il
      devra gérer deux ensembles de données à savoir l'état des
      clients et la charge des client.
        
      Les donnés seront récupéré par interrogation (polling) des
      esclaves par le maître. Chaque esclave recalcule sa charge à
      chaque fois que le maitre lui enverra une requête. Le polling se
      fera dès lors que le système aura besoin de données à jour.

      %\begin{itemize}
      %  \item polling de tout les esclaves lorsque le maître cherche à
      %          placer une nouvelle tâche.
      %  \item polling de tout les esclaves lors de l'élection d'un 
      %          nouveau maître.
      %  \item polling de tout les esclaves lorsque l'utilisateur
      %          demande l'état du système.
      %  \item polling du nouvel arrivant lors de l'insertion d'une
      %          nouvelle machine
      %\end{itemize}

    \subsection{Module de transfert}

      Afin de minimiser les échanges entre les différents sites, on
      laisse la possibilité aux esclaves d'exécuter leurs propres
      tâches sans passer par le maître. le transfert de tâche sera
      uniquement effectué lorsque l'escalve se considérera surchargé.
      Dans ce cas de figure il demandera au maître une machine
      destinatrice sur laquelle il pourra envoyer la tâche.
        
      \paragraph{Cas partage} C'est la politique de placement que nous
        implémenterons dans un premier temps. Chaque esclave possede
        un seuil qui lui est propre. Au départ le seuil sera fixe.
      
      \paragraph{Cas équilibrage} Cette politique sera mise en place
        dans un second temps. Le seuil sera calculé par le maître
        comme une moyenne des charges des esclaves.

    \subsection{Module de localisation}

      La décision de placement se fera au niveau du maître.
      Lorsqu'une nouvelle tâche est crée sur un esclave surchargé,
      celui-ci demande au maître l'identité d'une machine sur laquelle
      il peut envoyer cette tâche. Le maître interroge l'ensemble des
      esclaves pour obtenir leur charges puis il répond au site
      demandeur avec l'identité de la machine la moins chargé. Cette
      machine peut tout à fait être l'esclave ayant fait la
      demande. Une fois cette identité obtenue, la machine cliente
      communique la tâche à la machine cliente.
      
    \subsection{Tolérance aux pannes}

      La détection de panne d'une machine est réaliser lors du
      dépassement du délais fixé par un temporisateur. Le maître
      réalisera cette détéction en s'appuyant sur les messages de
      récolte de la charge des esclaves. Une fois la panne détecté, il
      signalera aux autres esclaves qu'une machine a brusquement
      quitté le système afin que les tâches qui étaient placées sur
      cette machine puissent être relancées.
      
      Un esclave détectera la panne du maître lors d'une demande de
      placement de tâche. Cette détéction initiera une éléction d'un
      nouveau maître. Le noeud élu endossera le rôle du maître afin de
      permettre au système de reprendre de manière transparente.

    \subsection{Intégration et retrait de machines}

      L'intégration d'un nouveau n\oe{}ud au placement se réalise comme
      suit. La machine fait parvenir son identité et sa charge au
      maître. Une fois que le maître a connaissance de la nouvelle
      machine il met à jour l'état global et envoie une confirmation
      à la machine demandeuse.
      
      La retrait d'un n\oe{}ud du placement se déroule
      ainsi. Lorsqu'un utilisateur demande le retirait de sa machine
      du système de placement réparti celle ci annonce au maître
      qu'elle n'accepte plus de tâche. Dès lors que l'execution de
      toute les tâches qu'elle comprend se termine elle annonce au
      maître qu'elle se retire, le maître mets alors à jour l'état
      global. Dans le cas d'un arret brutale le fonctionnement connu
      par un esclave est celui décrit en cas de faute franche.

  \newpage
\section{Modelisation avec UML}

  Afin de faciliter l'implementation nous avons décider de modéliser
  notre application en utilisant la méthode UML. Pour cela nous nous
  sommes appuyés sur les connaissances aquises lors du module
  Ingénierie Logiciel du semestre précédent.

  \subsection{Cas d'utilisation}
  
    \begin{figure}[h!]
      \centering
      \includegraphics[width=\textwidth]{img/analyse_DiagrammeCasUtilisation.pdf}
      \caption{Diagramme de cas d'utilisation}
    \end{figure}
    
    \subsubsection{Liste des acteurs}

      \begin{tabular}{|l|p{5cm}|l|}
        \hline \bf Acteur & \bf Description & \bf Type \\ \hline
        Utilisateur & Personne qui utilise une machine intégrée au
        placement système de placement réparti & Primaire \\ \hline
        Temporisateur & Indicateur de problème interne au système & Primaire, Secondaire
        \\ \hline
      \end{tabular}
    
  \subsubsection{Liste des cas d'utilisation}
    \begin{tabular}{|l|p{7cm}|}
      \hline
        \bf Cas d'utilisation &
        \bf Description \\
      \hline
        Intégrer une machine &
        Permet à un utilisateur d'intégrer sa machine dans le système
        de placement réparti \\
      \hline
        Exécuter un programme &
        Permet à un utilisateur d'exécuter un programme dans le
        système \\
      \hline
        Superviser le placement &
        Il permet à un utilisateur de consultés l'ensemble des données
        du placement réparti : machines, charges et programmes exécutés
        à distance \\
      \hline
        Retirer une machine &
        Il permet à un utilisateur de retirer sa machine du placement 
        réparti \\
      \hline
        Gérer les pannes &
        Il permet à un Timer de vérifier périodique si une machine n'est
        pas en panne et de la gérer \\
      \hline
    \end{tabular}
  \newpage

  \subsubsection{Cas Éxécuter un programme}

    \begin{figure}[h!]
      \centering
      \includegraphics[width=\textwidth]{img/analyse_DSCexecprog.pdf}
      \caption{Diagramme de séquence -- Cas Éxecuter un programme}
    \end{figure}

    Description des étapes permettant à un utilisateur d'exécuter un
    programme au sein du système de répartition de tâches. \\
    
    \noindent
    {\bf Acteur principal} Utilisateur \\
    {\bf Acteur secondaire} Aucun \\
    {\bf Séquence} Le cas d'utilisation commence lorsque l'utilisateur
      lance son programme \\
    {\bf Pré-conditions} \hfill
    \begin{itemize}
      \item Le placement doit être fonctionnel
      \item La machine doit être intégrée au placement réparti
      %\item Le programme doit être un programme valide au placement
    \end{itemize}
        
    \paragraph{Scénario nominal} La machine choisie est celle de
      l'utilisateur où la commande est lancée.

      \begin{enumerate}
        \item L'utilisateur demande le placement d'une tâche
        \item Le système choisit une machine parmi les machines du
            placement pouvant accueillir le programme
        \item Le système exécute le programme sur la machine choisie et attend le résultat
        \item Le système informe le client de la terminaison de son programme
      \end{enumerate}

    \paragraph{Alternatives}
      \begin{description}
        \item[A1] La machine choisie est une machine distante
          \begin{enumerate}
            \setcounter{enumi}{2}
            \item Le système enregistre le programme et ses paramètres dans
                   sa liste des programmes exécutés à distance
            \item Le système exécute le programme sur la machine choisie et
                   attend le résultat
            \item Le système retire le programme et ses paramètres de sa
                   liste des programmes exécutés à distance
          \end{enumerate}

        %% \item[A2] Le programme ou les paramètres sont incorrects
        %%   L'enchainement démarre après le point 2 de la séquence
        %%   nominale

        %%   \begin{enumerate}
        %%     \setcounter{enumi}{2}
        %%     \item Le système indique à l'utilisateur que les données
        %%       entrées du programme sont incorrectes La séquence
        %%       nominale reprend au début
        %%   \end{enumerate}
      \end{description}

   \paragraph{Exceptions}
     \begin{description}
       \item[E1] L'exécution du programme est interrompu suite à un
         problème quelconque du programme
         \begin{enumerate}
           \setcounter{enumi}{3}
           \item Le système indique à l'utilisateur de l'interruption
             de son programme
         \end{enumerate}

       \item[E2] Problème de communication entre machine source et cible
         \begin{enumerate}
           \setcounter{enumi}{3}
           \item Après trois tentatives sans acquittement, le système
             abandonne (la suite sera gérée par la gestion de pannes)
          \end{enumerate}
     \end{description}

  \subsubsection{Cas Intégrer une machine}

    \begin{figure}[h!]
      \centering
      \includegraphics[width=\textwidth]{img/analyse_DSseqintegermach.pdf}
      \caption{Diagramme de séquence -- Cas Intégrer une machine}
    \end{figure}

    Description des étapes permettant à un utilisateur d'intégrer sa
    machine au placement réparti afin qu'elle puisse répartir ou
    recevoir des programmes. \\

    \noindent
    {\bf Acteur principal} Utilisateur \\
    {\bf Acteur secondaire} Aucun \\
    {\bf Séquence} Le cas d'utilisation commence lors du démarrage du
      service de placement réparti \\ 
    {\bf Pré-conditions} \hfill
    \begin{itemize}
      \item Le placement doit être fonctionnel
      \item La machine diit pas être déjà intégrée au placement réparti
    \end{itemize}

    \paragraph{Scénario nominal} Intégration d'un esclave

      \begin{enumerate}
        \item L'utilisateur démarre le service du placement réparti
        \item Le système demande à tous les sites du placement celui qui représente le maitre
        \item Le système récupère l'identifiant fournit par le maitre
        \item Le système enregistre l'identifiant du maitre sur la machine de l'utilisateur
        \item Le système évalue la charge de la machine
        \item Le système enregistre l'identifiant et la charge de la machine dans le maitre
        \item Le système présente une invite de commandes
      \end{enumerate}

    \paragraph{Alternatives}
      \begin{description}
        \item[A1] La machine demandant l'identifiant du maître
          n'obitent pas de réponse
        \begin{enumerate}
          \setcounter{enumi}{2}
          \item Aucun site ne répond, à la suite de trois tentatives,
            le système lance une élection
          \item Le système initialise le nouveau maitre
          \item Le système évalue les charges de tous les machines du
            placement
        \end{enumerate}
      \end{description}

    \paragraph{Exception}
      \begin{description}
        \item[E1] La machine est déjà intégrée : une erreur se produit
          Post-conditions : Le système a enregistre ce qui suit
        \begin{itemize}
          \item L'identifiant et la charge locale de la machine
            intégrée 
          \item L'identifiant du maitre dans la machine intégrée
        \end{itemize}
      \end{description}

  \subsubsection{Cas Retirer une machine}

  \begin{figure}[h!]
    \centering
    \includegraphics[width=\textwidth]{img/analyse_DSseqretirerMaitre.pdf}
    \caption{Diagramme de séquence -- Cas Retirer la machine maître} 
  \end{figure}
  
  \begin{figure}[h!]
    \centering
    \includegraphics[width=\textwidth]{img/analyse_DSseqretireresclave.pdf}
    \caption{Diagramme de séquence -- Cas Retirer une machine esclave}
  \end{figure}

    \paragraph{Description}Décrire les étapes permettant à un 
      utilisateur de retirer sa machine du placement réparti.
      \begin{description}
        \item[Acteur principal] Utilisateur
        \item[Acteur secondaire] Aucun
        \item[Séquence] le cas d'utilisation commence lorsqu'un 
          utilisateur entre une commande de sortie (PAUSE, EXIT) du
          placement réparti sur sa machine
        \item[Pré-conditions] : La machine doit être une machine intégrée au placement
    \end{description}
    \paragraph{Scénario nominal} : Sortie d'un esclave sans forcer son arrêt
        \begin{enumerate}
          \item L'utilisateur entre la commande de sortie « PAUSE » et valide
          \item Le système vérifie la commande
          \item Le système marque la machine de l'utilisateur afin d'éviter l'exécution à distance de
              nouveaux programmes sur celle-ci
          \item Le système attend la terminaison des programmes exécutes à distance (en cours) sur la
              machine de l'utilisateur
          \item Le système extrait le données (identifiant, charge locale) de la machine des données du
              placement (données présentes sur le maitre)
          \item Le système revient à l'interface d'accueil du placement sur la machine de l'utilisateur
        \end{enumerate}
    \paragraph{Alternatives}
    A1 : Sortie d'un esclave en forçant son arrêt immédiat
            Enchainement démarre au début de la séquence nominale
        \begin{enumerate}
          \item L'utilisateur entre la commande de sortie « EXIT » et valide
          \item Le système vérifie la commande
          \item Le système arrête les programmes exécutés à distance sur la machine de l'utilisateur
          \item Le système extrait le données (identifiant, charge locale) de la machine des données du
              placement (données présentes sur le maitre)
          \item Le système relance sur chaque site, ses programmes interrompu qui s'exécutaient sur la
              machine retirée (voir cas d'utilisation : exécuter un programme)
        \end{enumerate}
    A2 : Sortie d'un maitre sans forcer son arrêt
        \begin{enumerate}
          \item L'utilisateur entre la commande de sortie « PAUSE » et valide
          \item Le système vérifie la commande
          \item Le système marque le maitre afin d'éviter l'exécution à distance de nouveaux programmes
              sur celle-ci
          \item Le système attend la terminaison des programmes exécutes à distance (en cours) sur le
              maitre de l'utilisateur
          \item Le système extrait le données (identifiant, charge locale) du maitre des données du
              placement (données présentes en locale)
          \item Le système élit un leader comme maitre parmi les esclaves
          \item Le système réinitialise le nouveau maitre élu
          \item Le système transfère les données du placement vers le nouveau maitre
          \item Le système informe les esclaves que le nouveau maitre est prêt à recevoir des requêtes
        \end{enumerate}
    A3 : Sortie d'un maitre en forçant son arrêt immédiat
         Enchainement démarre au début de la séquence nominale
        \begin{enumerate}
          \item L'utilisateur entre la commande de sortie « EXIT » et valide
          \item Le système vérifie la commande
          \item Le système arrête les programmes exécutés à distance sur le site maitre
          \item Le système extrait le données (identifiant, charge locale) du maitre des données du
              placement (données présentes en locale)
          \item Le système élit un leader comme maitre parmi les esclaves
          \item Le système réinitialise le nouveau maitre élu
          \item Le système transfère les données du placement vers le nouveau maitre
          \item Le système informe les esclaves que le nouveau maitre est prêt à recevoir des requêtes
          \item Le système relance sur chaque site, ses programmes interrompu qui s'exécutaient sur la
              machine retirée (voir cas d'utilisation : exécuter un programme)
        \end{enumerate}
    A4 : La commande entrée est invalide
         Enchainement démarre après le point 1de la séquence nominale, A1, A2 ou A3
        \begin{enumerate}
        \setcounter{enumi}{2}
          \item Le système indique à l'utilisateur que la commande est incorrectes
         La séquence nominale, A1, A2 ou A3 reprend au point 1
        \end{enumerate}
    A5 : la machine n'a pas de programme en cours d'exécution
         Enchainement 3, 4 de la séquence nominale, A2 ne sont plus considérés
    \paragraph{Exceptions}
    E1 : Problème communication entre machine source et cible
         L'enchainement démarre au point 5 de la séquence A1
            Après trois tentatives sans acquittement, le système abandonne (la suite sera gérée par la
            gestion de pannes)
    \paragraph{Post-conditions}
            Le système a retiré les informations suivantes :
                   - l'identifiant et la charge locale (retirés du placement réparti) de la machine retirée

  \subsubsection{Cas Superviser}

  \begin{figure}[h!]
    \centering
    \includegraphics[scale=0.45]{img/analyse_DSseqsuperviser.pdf}
    \caption{Diagramme de séquence -- Cas Superviser le placement réparti}
  \end{figure}

    \paragraph{Description} Décrire les étapes permettant à un utilisateur
      de visualiser les charges locales et les programmes exécutés à
      distance de tous les sites du placement afin de prendre des décisions
      (retrait, ajout de sa machine ou limiter le nombre de programmes
      distantes exécutés sur sa machine, ...).
      \begin{description}
        \item[Acteur principale] : Utilisateur,
        \item[Acteur secondaire] : Timer (traitement périodique)
        \item[Séquence] : le cas d'utilisation commence lorsque l'utilisateur entre la commande « ADMIN *»
        (Tous les sites) ou « ADMIN identifiant » (identifiant d'un site donnée participant au placement).
        \item[Pré-conditions] : le placement réparti de la machine de l'utilisateur doit être au moins démarré.
      \end{description}
    \paragraph{Scénario nominal}
        \begin{enumerate}
          \item L'utilisateur entre la commande « ADMIN * » pour avoir des informations sur tous les sites
             du placement.
          \item Le système récolte l'identifiant, la charge, les programme exécutés à distance de chaque site
             du placement réparti.
          \item Le système affiche l'identifiant, la charge, les programmes exécutés à distance de chaque site
             du placement.
          \item Le Timer maintient à jour la récolte et l'affichage périodiquement pour garder les données
             fraiches.
          \item L'utilisateur ferme l'affichage
          \item Le système revient à l'invite des commandes
        \end{enumerate}
    \paragraph{Alternatives}
    A1 : L'utilisateur entre la commande « ADMIN identifiant » pour avoir les informations d'une
    machine donnée.
         Enchainement démarre au début de la séquence nominale
         \begin{enumerate}
           \item L'utilisateur entre la commande « ADMIN identifiant » d'une machine
           \item Le système récolte l'identifiant, la charge, les programmes exécutés à distance de la machine
           \item Le système affiche les informations suivantes de la machine donnée :
               - son identifiant,
               - sa charge,
               - ses programmes exécutés à distance,
               - les programmes d'autres sites qu'il exécute
          \end{enumerate}
         La séquence nominale reprend au point 4
    \paragraph{Exceptions}
     E1 : la commande est « ADMIN identifiant » et le site n'est pas intégré au placement.
          Enchainement démarre après le point 1 de la séquence A1
          \begin{enumerate}
          \item le système indique l’erreur à l'utilisateur et revient à l'invite de commandes
          La séquence nominale reprend au début
          \end{enumerate}

  \subsubsection{Cas Gérer une panne}

  \begin{figure}[h!]
    \centering
    \includegraphics[scale=0.4]{img/analyse_DSseqpanneMaitre.pdf}
    \caption{Diagramme de séquence -- Cas Gestion de panne d'un maître}
  \end{figure}
  
  \begin{figure}[h!]
    \centering
    \includegraphics[scale=0.409]{img/analyse_DSseqpanneEsclave.pdf}
    \caption{Diagramme de séquence -- Cas Gestion de panne d'un esclave}
  \end{figure}

    \paragraph{Description} : Décrire les étapes permettant de gérer une machine du placement en panne détectée
    par un traitement périodique.
    \begin{description}
      \item[Acteur principal] Timer (Traitement périodique)
      \item[Acteur secondaire] Aucun
      \item[Séquence] le cas d'utilisation démarre lorsque la panne d'une machine est détectée.
      \item[Pré-conditions] la machine en panne doit être une machine intégrée au placement
    \end{description}
    \paragraph{Scénario nominale} : (Esclave) Avant de tomber en panne, le site avait des programmes exécutés à
    distance
        \begin{enumerate}
          \item Le Timer vérifie si une machine n'est pas en panne
          \item Le Timer détecte une panne d'une machine
          \item Le système vérifie si c'est un esclave ou le maitre en panne
          \item Le système extrait le données (identifiant, charge locale) de la machine des données du
              placement (données présentes sur le maitre)
          \item Le système demande aux autres sites d'arrêter ses programmes exécutant à distance
          \item Le système relance sur chaque machine, ses programmes interrompu sur la machine en
              panne (voir cas d'utilisation exécuter un programme)
        \end{enumerate}
    \paragraph{Alternatives}
    A1 : (Maitre) Avant de tomber en panne, le maitre avait des programmes exécutés à distance
         Enchainement démarre après le point 3 de la séquence nominale
        \begin{enumerate}
        \setcounter{enumi}{3}
          \item Le système demande aux autres sites d'arrêter les programmes du maitre exécutant à
              distance
          \item Le système élit un leader comme maitre parmi les esclaves
          \item Le système réinitialise le nouveau maitre élu
          \item Le système récolte et enregistre l'identifiant et la charge de chaque site du placement sur le
              maitre
          \item Le système relance sur chaque machine, ses programmes interrompu sur la machine maitre
              en panne (voir cas d'utilisation exécuter un programme)
        \end{enumerate}
    A2 : Avant de tomber en panne, la machine n'exécutait pas de programmes pour d'autres
         L'enchainement du point 6 de la séquence nominale ou 8 de la séquence A1 est ignoré
    A2 : Avant de tomber en panne la machine n'exécutait pas des programmes sur d'autres
         L'enchainement du point 5 de la séquence nominale ou 4 de la séquence A1 est ignoré
    A3 : Le Timer ne détecte pas de panne
       L'enchainement des points 3, 4, 5, 6, 7, 8 sont ignorés
    \paragraph{Post-conditions}
            La machine n'est plus dans le placement et le système a retiré les informations suivantes :
                – cas esclave : les programmes exécutés à distance du site en panne


%%%
\subsection{Données métier}
  
  \noindent Côté esclave :
    \begin{itemize}
      \item Le maître
      \item Les programmes exportés (ses programmes qui sont en train de s'exécuter à distance)
      \item Les programmes importés (les programmes des autres sites qui s'exécutent sur lui)
     \end{itemize}
        
   \noindent Côté maître :
     \begin{itemize}
       \item Tous les esclaves
       \item Les programmes exportés
       \item Les programmes importés
       \item La liste des sites soupçonnes d'être en panne
      \end{itemize} 
        
\subsection{Diagramme de classe}
  Etant donnée que notre projet a pour contrainte d'étre écrit en C ce 
  diagramme représente les structures de données à définir dans notre
  application.

     \begin{figure}[h!]
       \centering
       \includegraphics[width=\textwidth]{img/analyse_dcl.pdf}
       \caption{Diagramme de classe}
     \end{figure}
\newpage

\subsection{Diagramme de composant}

  \begin{figure}[h!]
    \centering
    \includegraphics[angle=270,width=0.8\textwidth]{img/analyse_DiagrammeComposant.pdf}
    \caption{Diagramme de composant}
  \end{figure}

  % Part 3:
\section{Legal issues}

  \begin{frame}
    \tableofcontents[currentsection, hideothersubsections]
  \end{frame}

  %%
  %% Genesis
  %%
  \subsection{The Starting Point}
    \begin{frame}
      \frametitle{\subsecname}
      \framesubtitle{Genesis}

      % At this time file-sharing was primitive
      % Low-seed Internet connection through 56K modems
      % MP3 a recent file format (compressed music)
      % Exchanged on newsgroup like USENET
      % That is to say not suitable to newbies
      % (But not P2P)
      \begin{block}{Genesis [1:1]}
        In the beginning God created the \textbf{low-speed Internet}
        and the \textbf{\textsc{MP3} file-format}\footnote{My revision
          of King James version}. 
      \end{block}
    \end{frame}

    \begin{frame}
      \frametitle{\subsecname}
      \framesubtitle{Genesis}

      % First wide P2P file-sharing project
      % One week after its launch Napster has been downloaded 15 thousand times
      % On July 2000 Napster has been downloaded more than 23 million times
      % Napster had a weakness: Centralized peer/file search
      % Napster spread MP3 and file-sharing to any Internet user
      {\large\bf Napster}
      \begin{description}[XXXX]
        \item[1999] Launch \& RIAA lawsuit
        \item[2000] Metallica \& Dr. Dre lawsuit
        \item[2001] Legal shutdown
      \end{description}
    \end{frame}
    
  %%
  %% Exodus
  %%
  \subsection{Decentralization}
    \begin{frame}
      \frametitle{\subsecname}
      \framesubtitle{Exodus}

      % (I dit it for the Lulz)
      % Decreasing Internet access price
      % Increasing Internet access speed
     \begin{block}{Exodus [1:1]}
        These are the names of the sons of \textbf{Napster} who went
        to \textbf{peer-to-peer} with \textbf{decentralization}.
      \end{block}
    \end{frame}

    \begin{frame}
      \frametitle{\subsecname}
      \framesubtitle{Exodus}

      % KaZaA has been downloaded 342 million times 
      % KaZaA had a strength: Decentralized peer/file search and thus
      %   not responsible!
      % BUMA/STEMRA is the Netherlands's SACEM
      {\large\bf KaZaA}
      \begin{description}[XXXX]
        \item[2001] Launch \& \textsc{BUMA/STEMRA} lawsuit
        \item[2002] \textbf{Not guilty}
      \end{description} \pause
      \vspace{1em}

      % eMule has been download 639 million times
      % eMule had a strength: Chunked files. As soon as a chunk was
      %   downloaded it was shared
      % And alsa 3 to 4 million users & about 4 billion files
      % User were directly sued by RIAA & MPAA in the USA!
      % Full of fake & malwares
      {\large\bf eMule}
      \begin{description}[XXXX]
        \item[2002] Launch \& \textbf{not even sued}
      \end{description} \pause
      \vspace{1em}

      % BitTorrent have a strength: Optimized bandwith use &
      %   cutting edge technologies
      % Stills good in 2013
      {\large\bf BitTorrent}
      \begin{description}[XXXX]
        \item[2002] Launch
      \end{description}
    \end{frame}

  %%
  %% Numbers
  %%
  \subsection{Repression \& Results}
    \begin{frame}
      \frametitle{\subsecname}
      \framesubtitle{Numbers}

      % I dit it for the Lulz
      \begin{block}{Numbers [1:2]}
        Take a census of the whole \textbf{convicted person} by their
        clans and families, listing every man by name, one by one.
      \end{block}
    \end{frame}

    \begin{frame}
      \frametitle{\subsecname}
      \framesubtitle{Numbers}

      % What's my name?
      \begin{figure}
        \centering
        \includegraphics[scale=1.7]{img/P3-Hadopi.jpg}
        \caption{Beware, this is copyrighted material}
      \end{figure}
    \end{frame}

    \begin{frame}
      \frametitle{\subsecname}
      \framesubtitle{Numbers}

      % Hadopi have a weakness: They don't know how it works
      % One sentence, one (Rihanna) song, one ex-wife of him!
      {\large\textbf{Hadopi} \footnotesize -- The French touch}
      \begin{description}[XXXX]
        \item[2009] Launch \pause
        \item[2012] \textbf{One sentence} of \texteuro 150 fine \pause 
        \item[2013] \alert{\textbf{\texteuro 30 million spent!}}
      \end{description} \pause
      \vspace{1em}

      % Direct download & Minitel 2.0
      % Unique french visitor
      {\large\textbf{Minitel 2.0} \footnotesize -- Megaupload}
      \begin{description}[XXXX]
        \item[2008] 350 thousand visitors a month \pause
        \item[2010] 7,4 million visitors a month
      \end{description}
    \end{frame}

  %%
  %% Deuteronomy
  %%
  \subsection{Possible Evolutions}
    \begin{frame}
      \frametitle{\subsecname}
      \framesubtitle{Deuteronomy -- Promised Land}

      % I dit it for the Lulz
      \begin{block}{Deuteronomy [1:8]}
        Go in and take possession of the land the
        \textbf{\textsc{HADOPI} didn't know} he would give to your
        fathers and to their descendants after them.
      \end{block}
    \end{frame}

    \begin{frame}
      \frametitle{\subsecname}
      \framesubtitle{Deuteronomy -- Promised Land}

      % Eventually laws and trials about file-sharing through the 00's
      %   acted as catalysers for the evloution of P2P technologies
      % It's predictable that generalized anonimity & ciphering is
      %   the future of P2P as possibly private P2P is
      % (P3P) It almost already exist avec F2F through VPN or Darknets
      %   as Freenet
      {\large\bf My prophecies}
      \begin{itemize}
        \item {\color<2->{lightgray} Pay?} \pause
        \item Anonimity
        \item Ciphered
        \item Private or trusted
      \end{itemize}
    \end{frame}


\end{document}

%% Exigences du rapport 
 %% Le Rapport de conception doit contenir :
  % - Définition du sujet
  % - Compréhension des problèmes
  % - Présentation des choix effectués
  % - Interface utilisateur
  % - Découpage en module
  % - Architecture globale du système (spécification des interfaces
  %   entre modules, spécification des différents messages échangés
  %   entre les processus répartis).
  % La solution doit être modulaire (i.e. composée de modules regroupant
  % des entités de même nature) et extensible (i.e. devant permettre
  % l'intégration de fonctionnalités supplémentaires.
  %
 %% Le rapport final reprendra le rapport de conception (corrigé)
  % + mise en œuvre (réalisations) + manuels instal/utilisateur
  % + jeu de tests
  % + pblms rencontrés, etc.)

%% Nota Bene :
 %  - Bien distinguer la partie qui developpe sur les politiques À
 %    DERTERMINER et celle qui parle des modules définissant des
 %    méthodes concrètes de réalisation.
 %  - Pas de politique de seléction pour le placement réparti
 %    dynamique dans la thèse de B. Folliot.
 %  - Le calcul de la charge ne se fait pas en fonction de
 %    l'utilisation CPU parce que le multitasking ça existe !
