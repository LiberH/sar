\documentclass{article}

\usepackage{lmodern}
\renewcommand*\familydefault{\sfdefault}
\usepackage[T1]{fontenc}
\usepackage[utf8]{inputenc}
\usepackage[francais]{babel}
\usepackage{tikz}
\usetikzlibrary{arrows}

\usepackage{amsthm}
\newtheorem{lemme}{Lemme}

\usepackage{listings}
\lstset{
  numbers=left,
  escapeinside={(*}{*)},
  basicstyle=\footnotesize,
  columns=flexible,
  breaklines=true,
  frame=single,
  literate={é}{{\'e}}1 {è}{{\`e}}1 {à}{{\`a}}1,
  morekeywords={si,alors,sinon},
}

\title{
  {\bf Algorithmique Répartie Avancée} \\
  Rapport de projet}
\author{
  Armel   \textsc{Mangean} -- \oldstylenums{3262313} \\
  Idrissa \textsc{Sokhona} -- \oldstylenums{3101058}}
\date{Octobre 2013}

\begin{document}

  \maketitle

  \section{Algorithme}
    Nous considérons un algorithme probabiliste d'exploration de cylindres
    carrés {\bf de côté $S$ impair et superieur ou égal à 7}.

    Cet algorithme requière {\bf exactement $k$ = 4 robots} doués de détection faible
    de multiplicité. 

    \subsection{$Z$-configuration}
      La {\bf Phase Set-Up} se termine lorsque, depuis une configuration
      quelconque, on atteint une $Z$-configuration.  La figure
      \ref{fig:z-configuration} donne une représentation d'une $Z$-configuration
      pour un cylindre carré de côté $S = 7$. \footnote{Nous avons pris la
        libérté d'aider certains de nos camarades en partageant les sources
        \LaTeX{} de ce schéma.} \medskip

      \begin{figure}[h]
        \centering
        \begin{tikzpicture}
          \draw[ultra thick] (0,4) -- (8,4);
          \foreach \i in {1,...,7} {
            \draw[dashed]           (\i,1) -- (\i,7);
            \draw[dashed]           (0,\i) -- (8,\i);
            \draw[thick,fill=white] (8,\i) circle (1mm) {};
            \draw[thick,fill=black] (0,\i) circle (1mm) {};
          }

          \node at (3,3) [circle,draw,ultra thick,fill=white] {};
          \node at (3,4) [circle,draw,ultra thick,fill=white] {};
          \node at (4,4) [circle,draw,ultra thick,fill=white] {};
          \node at (4,5) [circle,draw,ultra thick,fill=white] {};
        \end{tikzpicture}
        \caption{Une $Z$-configuration sur un cylindre carré de côté 7.}
        \label{fig:z-configuration}
      \end{figure}

      Une configuration $C$ est appelée $Z$-configuration si est seulement si
      les trois conditions suivantes sont satisfaites :
        \begin{enumerate}
          \item $C$ ne contient aucune tour,
          \item Exactement deux robots sont localisés sur l'anneau central, et
          \item Deux robots sont localisés sur deux n\oe{}uds voisins
            de l'anneau central tel que chacun des robots {\bf sur
              l'anneau central} aient exactement deux robots dans leur
              voisinnage.
        \end{enumerate} \medskip

      Afin que les deux robots localisés sur l'anneau central aient chacun
      exactement deux robots dans leur voisinnage il est nécessaire que ceux-ci
      soient voisins entre-eux.

    \subsection{Phase Tour}

      La phase tour est décrite dans l'algorithme \ref{lst:phasetower}. Partant
      d'une $Z$-configuration le système finira par atteindre une configuration
      où une tour est formée. Dans ce but nous procédons comme suit. Soient
      $R_1$ et $R_2$ les deux robots localisés sur l'anneau central. $R_1$ et
      $R_2$ essaient de bouger l'un vers l'autre. Finalement, seul un de ces
      robots bougera et la tour sera formée comme prouvé dans les deux lemmes
      suivants.

%%%
\lstset{title={Algorithme \ref{lst:phasetower} {\it Phase Tour}.}}
\begin{lstlisting}[label=lst:phasetower]
  si je suis localisé sur l'anneau central alors
    essai de déplacement vers le noeud voisin de l'anneau central occupé par un robot
\end{lstlisting}
%%%

      \begin{lemme}\label{lem:tower}
        Soit $\gamma$ une $Z$-configuration. Si $\gamma$ est la configuration à
        l'instant $t$, alors la configuration à l'instant $t + 1$ est soit
        identique à $\gamma$, soit la configuration contenant une tour.
      \end{lemme}

      \paragraph{Preuve.} Soient $R_1$ et $R_2$ les robots localisés sur l'anneau
        central dans $\gamma$. Dans $\gamma$ tous les robots executent
        l'algorithme \ref{lst:phasetower}. Donc, depuis $\gamma$ seul $R_1$ et
        $R_2$ peuvent bouger : $R_1$ peut bouger vers $R_2$ et inversement
        (cf. algorithme \ref{lst:phasetower}). Quand l'un ou les deux robots
        bougent on obtient soit une configuration contenant une tour soit une
        $Z$-configuration identique à $\gamma$ et le lemme tient. \qed

      \begin{lemme}
        Depuis une $Z$-configuration, le système atteint une configuration
        contenant une tour en un temps fini.
      \end{lemme}

      \paragraph{Preuve.} Par le lemme \ref{lem:tower}, nous savons que partant
        d'une $Z$-configuration $\gamma$, le système soit reste dans la même
        configuration soit atteint une configuration contenant une tour.

        Soient $R_1$ et $R_2$ les deux robots localisé sur l'anneau
        central. Seul $R_1$ et $R_2$ peuvent (de manière probablistique) décider
        de bouger dans $\gamma$. De plus, par la propriété d'équité ({\it
          fairness}), l'un ou les deux robots seront finalement activés. Malgrès
        le choix de l'ordonnanceur, il y a une probabilité strictement positive
        qu'un seul des robots decide de manière probabiliste de bouger. Dans ce
        cas le système atteint une configuration contenant une tour
        (cf. algorithme \ref{lst:phasetower}). \qed \medskip

      Nous considérerons dans la suite de ce document une configuration issue de
      la phase tour comme une configuration de type {\it Tour}.

    \subsection{Phase Orientation}
      \label{sec:orientation}

      Une orientation du cylindre carré est determinée de la manière qui
      suit. La configuration considérée contient une tour $T$ formée de deux
      robots ainsi qu'un troisième robot $u$ distinct des deux premiers et
      localisé sur un autre noeud. Nous pouvons alors déterminer un système de
      coordonnées où chaque n\oe{}ud a des coordonnées uniques (cf. figure
      \ref{fig:orientation}). La tour $T$ définie l'origine du repère et a donc
      pour coordonnées (0,0). Considérant le plus petit rectangle englobant
      $R_{Tu}$, on peut attribuer à $u$ les coordonnées
      ($\alpha_{Tu}$,$\beta_{Tu}$). L'abscisse est donné par le vecteur reliant
      le n\oe{}ud $T$ au n\oe{}ud de coordonnées ($\alpha_{Tu}$,0). L'ordonnée
      est donné par le vecteur reliant le n\oe{}ud $T$ au n\oe{}ud de
      coordonnées (0,$\beta_{Tu}$).

      Afin, de pouvoir différencier l'abscisse de l'ordonnée il est nécessaire
      de pouvoir différencier $\alpha_{Tu}$ de $\beta_{Tu}$. Pour ce faire on
      considère le plus petit rectangle englobant $R_{Tu}$ tel que $\alpha_{Tu}$
      soit strictement inferieur à $\beta_{Tu}$. \medskip

      \begin{figure}[h]
        \centering
        \begin{tikzpicture}
          \draw[fill=lightgray] (3,4) rectangle (4,6);
          \node at (3,6) [below right] {$R_{Tu}$}; 
          \draw[ultra thick] (0,4) -- (8,4);
          \foreach \i in {1,...,7} {
            \draw[dashed]           (\i,1) -- (\i,7);
            \draw[dashed]           (0,\i) -- (8,\i);
            \draw[thick,fill=white] (8,\i) circle (1mm) {};
            \draw[thick,fill=black] (0,\i) circle (1mm) {};
          }

          \newcount\x \newcount\y
          \foreach \i in {1,...,7} {
            \foreach \j in {1,...,7} {
              \x=\i \advance\x by -3
              \y=\j \advance\y by -4
              \node at (\i,\j) [above right] {\tiny (\the\x,\the\y)};
            }
          }

          \node at (3,3) [circle,fill=white,draw,ultra thick]       {};
          \node at (3,4) [circle,fill=white,draw,ultra thick]       {};
          \node at (3,4) [circle,fill=white,draw,ultra thick,right] {};
          \node at (4,6) [circle,fill=white,draw,ultra thick]       {};
        \end{tikzpicture}
        \caption{Une configuration de type tour sur un cylindre carré de côté 7.}
        \label{fig:orientation}
      \end{figure}

      Nous considérerons dans la suite de ce document une configuration issue de
      la phase tour comme une configuration de type {\it Orienté}.

    \subsection{Phase Exploration}

      La première phase de l'algorithme d'exploration (cf. algorithme
      \ref{lst:phaseexplo1}) nécessite les trois conditions indiquées
      ci-dessous.

      \begin{enumerate}
        \item L'orientation construite par le robot $\ast$ doit être
          non-ambigüe,
        \item Robot $\circ$ doit être capable de visiter les deux carrés centrès
          sur la tour, et
        \item Robot $\ast$ doit être capable de se décaler d'au moins trois
          lignes de la tour.
      \end{enumerate}

      \paragraph{Preuve.}
        \begin{enumerate}
          \item La construction d'une orientation non-ambigüe nécessite
            l'utilisation d'un plus petit rectangle englobant $R_{Tu}$ tel que
            $0 < \alpha_{Tu} < \beta_{Tu} \geq 2$.
            
          \item Lorsque Robot $\circ$ visite le second carré centré sur la tour
            il se décale d'au moins deux lignes de la tour. Il génère alors un
            plus petit rectangle englobant $R_{T\circ}$ tel que $\beta_{T\circ}
            \geq 2$.

          \item Lorsque Robot $\ast$ se décale d'au moins trois lignes de la
            tour il génère le plus petit rectangle englobant $R_{T\ast}$ tel que
            $\beta_{T\ast} \geq 3$.

        \end{enumerate} \medskip

        Ces trois conditions réunies impliquent $S \geq 7$. \qed

%%%
\lstset{title={Algorithme \ref{lst:phaseexplo1} {\it Phase Exploration 1}.}}
\begin{lstlisting}[label=lst:phaseexplo1]
  si (*$R_{T\ast}$*) est un (1, 1)-rectangle alors
    si je suis (*$\ast$*) alors
      déplacement dans la direction opposée à (*$\circ$*)

  sinon soit (*$S_\ast$*) le système de coordonnées construit par (*$R_{T\ast}$*)
    si (*$R_{T\ast}$*) est un (1, 2)-rectangle alors
      si (*$\circ = (-1, -1)$*) alors
        si je suis (*$\ast$*) alors déplacement en (1, 3)
      sinon si je suis (*$\circ$*) alors exécuter procédure Carré(*$(1)$*)

    sinon si (*$R_{T\ast}$*) est un (1, 3)-rectangle alors
      si (*$\circ = (-1, -1)$*) alors
        si je suis (*$\ast$*) alors déplacement en (2, 3)
      sinon si (*$\circ = (-2, -2)$*) alors
        si je suis (*$\circ$*) alors déplacement en (*$(-2, -1)$*)

    sinon si (*$R_{T\ast}$*) est un (2, 3)-rectangle alors
      si (*$\circ = (-2, -2)$*) alors
        si je suis (*$\ast$*) alors déplacement en (1, 3)
      sinon si je suis (*$\circ$*) alors exécuter procédure Carré(*$(2)$*)
\end{lstlisting}
%%%

      \begin{lemme}
        Soit $\gamma$ une configuration de type Tour. Si $\gamma$ est la
        configuration à l'instant $t$ alors la configuration à l'instant $t + 1$
        est une configuration de type Orienté.
      \end{lemme}
      
      \paragraph{Preuve.} Soient $\ast$ ($\circ$) le robot le plus loin
        (resp. proche) de $T$ dans $\gamma$. Dans $\gamma$ tous les robots 
        executent l'algorithme \ref{lst:phaseexplo1}. Seul $\ast$ peut bouger,
        il se déplace dans la direction opposé à $\circ$ (cf. algorithme
        \ref{lst:phaseexplo1}). On obtient alors une configuration de type
        Orienté et le lemme tient. \qed \medskip

      Le comportement de la second phase de l'algorithme d'exploration est
      décrite par la figure \ref{fig:phaseexplo2}.

      \begin{figure}[h]
        \centering
        \tikzstyle{fleche} = [->,>=stealth',shorten >=5pt,thick]
        \begin{tikzpicture}[scale=0.7, transform shape]

          \draw[ultra thick] (0,8) -- (16,8);
          \foreach \i in {1,...,15} {
            \draw[dashed]           (\i, 1) -- (\i,15);
            \draw[dashed]           ( 0,\i) -- (16,\i);
            \draw[thick,fill=white] (16,\i) circle (1mm) {};
            \draw[thick,fill=black] ( 0,\i) circle (1mm) {};
          }

          \node at (6,9) [circle,fill=white,draw,ultra thick]       {};
          \node at (8,8) [circle,fill=white,draw,ultra thick]       {};
          \node at (8,8) [circle,fill=white,draw,ultra thick,right] {};

          \node at (9,5) [circle,fill=black,draw,ultra thick]       {};
          \node at (8.5,5) [above] {\bf 1};
          \draw[thick]  ( 9, 5) to ( 5, 5);
          \draw[thick]  ( 5, 5) to ( 5,11);
          \draw[thick]  ( 5,11) to (11,11);
          \draw[fleche] (11,11) to (11, 5);

          \newcount\a
          \newcount\b
          \newcount\c
          \foreach \i in {0,...,3} {
            \a 11 \advance \a by  \i
            \b  5 \advance \b by -\i
            \node at (\a,\b) [circle,fill=black,draw,ultra thick] {};

            \c 2 \advance \c by \i
            \node at (\a,\b-.5) [left] {\bf\the\c};
            \draw[thick] (\a,\b) to (\a,\b-1);

            \advance \b by -1
            \draw[thick]  (\a,\b) to (\b,\b); \advance \a by 1
            \draw[thick]  (\b,\b) to (\b,\a); 
            \draw[thick]  (\b,\a) to (\a,\a);
            \draw[fleche] (\a,\a) to (\a,\b);
          }

          \node at (15,1) [circle,fill=black,draw,ultra thick] {};
        \end{tikzpicture}
        \caption{Seconde phase d'exploration.}
        \label{fig:phaseexplo2}
      \end{figure}

    \subsection{Phase Set-Up}
      \subsubsection{Discrimination}

        \begin{figure}[h]
          \centering
          \tikzstyle{fleche} = [->,>=stealth',shorten >=5pt,thick]
          \begin{tikzpicture}
            \draw[ultra thick] (0,4) -- (8,4);
            \foreach \i in {1,...,7} {
              \draw[dashed]           (\i,1) -- (\i,7);
              \draw[dashed]           (0,\i) -- (8,\i);
              \draw[thick,fill=white] (8,\i) circle (1mm) {};
              \draw[thick,fill=black] (0,\i) circle (1mm) {};
            }
            
            \draw[fleche] (2,5) to (2,6); \node at (2,5.5) [left] {\bf try};
            \draw[fleche] (7,5) to (7,6); \node at (7,5.5) [left] {\bf try};
            \node at (2,5) [circle,draw,ultra thick,fill=white] {};
            \node at (2,6) [circle,draw,ultra thick,fill=black] {};
            \node at (4,1) [circle,draw,ultra thick,fill=black] {};
            \node at (5,2) [circle,draw,ultra thick,fill=black] {};
            \node at (7,5) [circle,draw,ultra thick,fill=black] {};
          \end{tikzpicture}
          \caption{Un exemple de discrimination pour la phase Set-Up.}
          \label{fig:discri}
        \end{figure}

        L'algorithme réalisant la phase Set-Up présenté ici nécessite à chaque
        étape la discrimination d'un unique robot parmi les $k$ y
        participant. Ce dernier est déterminé comme étant le robot ayant la
        distance minimale non nulle à l'anneau central. Si plusieurs robots
        partagent cette même distance minimale on réalise un déplacement
        probabiliste sur la chaine jusqu'à obtenir un unique robot ayant la
        distance minimale à l'anneau central. La figure \ref{fig:discri} donne
        un exemple d'une telle discrimination.

      \subsubsection{Principe}
      
        Ayant un robot discriminé à chaque itération de l'algorithme on procéde
        au placement de chacun des $k$ robots les uns après les autres jusqu'à
        obtenir une $Z$-configuration. La figure \ref{fig:principe} donne un
        exemple de déroulement de cet algorithme. Ce dernier peut-être
        brièvement résumé par ce qui suit :

        \begin{enumerate}
          \item Placer le $1^{er}$ robot sur l'anneau central
          \item Placer le $2^{nd}$ robot sur l'anneau central dans le voisinage
            du $1^{er}$
          \item Placer le $3^{eme}$ robot dans le voisinage du $1^{er}$ à
            distance $1$ de l'anneau central
          \item Placer le $4^{eme}$ robot dans le voisinage du $2^{nd}$ à
            distance $1$ de l'anneau central, du côté opposé au $3^{eme}$ robot
            par rapport à l'anneau central
        \end{enumerate}

        \begin{figure}[h]
          \centering
          \tikzstyle{fleche} = [->,>=stealth',shorten >=5pt,thick]
          \begin{tikzpicture}
            \draw[ultra thick] (0,4) -- (8,4);
            \foreach \i in {1,...,7} {
              \draw[dashed]           (\i,1) -- (\i,7);
              \draw[dashed]           (0,\i) -- (8,\i);
              \draw[thick,fill=white] (8,\i) circle (1mm) {};
              \draw[thick,fill=black] (0,\i) circle (1mm) {};
            }
            
            \draw[fleche] (7,5) to (7,4); \node at (7,4.5) [left] {\bf 1};
            \node at (7,5) [circle,draw,ultra thick,fill=white] {};
            \node at (7,4) [circle,draw,ultra thick,fill=black] {};

            \draw[thick]  (2,6) to (2,4); \node at (2,5.5) [left] {\bf 2};
            \draw[fleche] (2,4) to (6,4);
            \node at (2,6) [circle,draw,ultra thick,fill=white] {};
            \node at (6,4) [circle,draw,ultra thick,fill=black] {};

            \draw[thick]  (5,2) to (7,2); \node at (5.5,2) [above] {\bf 3};
            \draw[fleche] (7,2) to (7,3);
            \node at (5,2) [circle,draw,ultra thick,fill=white] {};
            \node at (7,3) [circle,draw,ultra thick,fill=black] {};

            \draw[thick]  (4,1) to (4,5); \node at (4,1.5) [left] {\bf 4};
            \draw[fleche] (4,5) to (6,5);
            \node at (4,1) [circle,draw,ultra thick,fill=white] {};
            \node at (6,5) [circle,draw,ultra thick,fill=black] {};
          \end{tikzpicture}
          \caption{Un exemple de déroulement de l'algorithme de la phase Set-Up.}
          \label{fig:principe}
        \end{figure}

    \subsection{Borne inférieure}

      Nous rappelons que l'ordonnanceur est dirigé par un {\it adversaire}.

      \begin{figure}[h]
        \centering
        \begin{tikzpicture}
          \draw[ultra thick] (0,4) -- (8,4);
          \foreach \i in {1,...,7} {
            \draw[dashed]           (\i,1) -- (\i,7);
            \draw[dashed]           (0,\i) -- (8,\i);
            \draw[thick,fill=white] (8,\i) circle (1mm) {};
            \draw[thick,fill=black] (0,\i) circle (1mm) {};
          }

          \node at (3,2) [circle,draw,ultra thick,fill=white] {};
          \node at (6,2) [circle,draw,ultra thick,fill=white] {};
          \node at (3,6) [circle,draw,ultra thick,fill=white] {};
          \node at (6,6) [circle,draw,ultra thick,fill=white] {};
        \end{tikzpicture}
        \caption{Une configuration de départ symétrique avec 4 robots et $S = 7$.}
        \label{fig:k4sym}
      \end{figure}

      \begin{lemme}
        Soit $\gamma$ une configuration de départ symétrique. Si $\gamma$ est la
        configuration à l'instant $t$ alors la configuration à l'instant $t + 1$
        est une configuration symétrique.
      \end{lemme}
      
      \paragraph{Preuve.} Soient $R_i$ les $k$ robots dans $\gamma$. Dans
        $\gamma$ tous les robots peuvent bouger. $\gamma$ étant symétrique les
        robots réalisent tous le même déplacement respectivement à leur place
        d'origine. On obtient alors une configuration symétrique et le lemme
        tient. \qed \medskip

      Il est possible de trouver des configurations de départ symétrique pours
      $k = 4$ robots (cf. figure \ref{fig:k4sym}) et $k = 5$ (cf. figure
      \ref{fig:k5sym}). Il n'existe donc pas d'algorithme déterministe
      permettant de réaliser l'exploration d'un cylindre carré avec $k = 4$ ou
      $k = 5$ robots.

      \begin{figure}[h]
        \centering
        \begin{tikzpicture}[scale=0.7, transform shape]
          \draw[ultra thick] (0,2) -- (16,2);
          \foreach \i in {1,...,3} {
            \draw[dashed]           ( 0,\i) -- (16,\i);
            \draw[thick,fill=white] (16,\i) circle (1mm) {};
            \draw[thick,fill=black] ( 0,\i) circle (1mm) {};
          }
          \foreach \i in {1,...,15} {
            \draw[dashed] (\i,0) -- (\i,4);
          }

          \node at ( 1,2) [circle,draw,ultra thick,fill=white] {};
          \node at ( 4,2) [circle,draw,ultra thick,fill=white] {};
          \node at ( 7,2) [circle,draw,ultra thick,fill=white] {};
          \node at (10,2) [circle,draw,ultra thick,fill=white] {};
          \node at (13,2) [circle,draw,ultra thick,fill=white] {};
        \end{tikzpicture}
        \caption{Une configuration de départ symétrique avec 5 robots et $S = 15$.}
        \label{fig:k5sym}
      \end{figure}

    \medskip
    Nous avons pu déterminer dans ce document que la construction d'une
    orientation non-ambigüe nécessite exactement 3 robots
    (cf. section \ref{sec:orientation}). Il n'est donc pas possible de réaliser une
    exploration d'un cylindre carré que ce soit de manière déterministe ou
    probabiliste avec $k = 3$ robots.

\end{document}
