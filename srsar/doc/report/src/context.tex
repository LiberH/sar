\chapter{Contexte}
  \section{Scolaire}
    
    Ce stage s'inscrit dans mon cursus scolaire en tant que stage de fin
    d'étude. J'arrive effectivement à la fin du Master que j'ai eu la chance de
    réaliser à l'Université Pierre et Marie Curie. J'y ai effectué un Master
    Informatique dans la filière Systèmes et Applications Réparties. En fin de
    seconde année j'ai opté pour l'option Systèmes Répartis Embarqués et Temps
    Réel. C'est dans cette voie que s'inscrit mon stage puisqu'il se situe dans
    le domaine du temps réel embarqué.

    ~
    
    Ce stage a une orientation recherche au sens où il requiert une étude de
    divers travaux de l'état de l'art dans la modèlisation des systèmes temps
    réel. Il s'inscrit dans un projet de recherche académique et a pour objet
    la mise en \oe uvre de résultats de recherches théoriques.

    ~
    
    Ce stage m'ouvre la porte vers le monde de la recherche où j'aimerais
    entamer ma carrière professionnelle. Ce stage est en effet la dernière étape
    dans le projet scolaire qui a été le mien depuis mon entrée en Institut
    Universitaire de Technologie en Informatique à Nantes. C'est alors que j'ai
    construit mon ambition de travailler dans la recherche publique et que j'ai
    rencontré les enseignants qui m'ont suivit jusqu'à ce jour dans mon
    parcours.
    
  \section{Professionnel}
    \subsection{L'IRCCyN}

      Il m'a été donné l'opportunité de réaliser mon stage à l'IRCCyN,
      l'Institut de Recherche en Communication et Cybernetique de Nantes. Cet
      établissement est une Unité Mixte de Recherche, du Centre National pour la
      Recherche Scientifique qui a pour ambition d'étendre l'ensemble des
      connaissances actuelles autour de ce qui concerne l'interaction entre les
      systèmes, la cybernétique.

      ~
    
      Cela représente un large champs d'étude et de nombreux domaines y sont
      donc étudiés. Parmis ces domaines on peut en autres trouver la logistique,
      la psychologie cognitive, la bioinformatique, l'automatique, le traitement
      du signal, la robotique, etc.
        
      ~
    
      Mon stage a débuté le 7 Avril 2014 et prendra fin le 12 Septembre
      2014. Une fermeture administrative de 2 semaine en début Août m'aura
      permis de profiter de cet été 2014.

    \subsection{L'équipe Systèmes Temps Réel}

      Au sein de cet ensemble de domaines d'étude se trouve l'informatique et
      plus précisement l'informatique temps réel. Elle définit des méthodes et
      techniques nécessaires à la mise en place de systèmes informatiques
      critiques. Ici, le terme critique dénote d'une part un besoin important
      en fiabilité, autrement dit il s'agit de systèmes ne devant pas
      défaillir, et d'autre part de certaines contraintes liées à un respect
      d'échéances temporelles.

      ~
    
      On peut prendre pour exemple le système informatique intégré aux voitures
      modernes permettant d'améliorer le confort de conduite et surtout la
      sécurité des passagers par un contrôle informatisé du véhicule. Cela est
      permis par l'identification des situations dangereuses et les réponses qui
      sont alors appliquées dans des délais brefs.

      ~
    
      Au fil des années ces sytèmes deviennent de plus en plus performants du
      point de vue de l'usager mais bien évidemment de plus en plus complexes du
      point de vue du concepteur, demandant donc toujours plus de travail pour
      les construire. C'est le rôle de l'équipe Systèmes Temps Réel de l'IRCCyN,
      au sein de laquelle j'effectue mon stage, de produire des outils
      facilitant la conception et l'analyse de tels systèmes.

