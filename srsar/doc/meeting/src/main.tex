\documentclass{beamer}

\usepackage[utf8]{inputenc}
\usepackage[T1]{fontenc}
\usepackage[francais]{babel}
%\usepackage{textcomp}

%\usetheme{Rochester}
\setbeamertemplate{navigation symbols}{}
\setbeamertemplate{footline}[frame number]

\title{}
\author{Armel}
\institute{}
\date{Mai 2014}

\begin{document}

  \begin{frame}
    \frametitle{Modèle AADL}
    \framesubtitle{Implémentation matérielle}

    NXT
    \begin{itemize}
      \item Capteur de distance
      \item Roue
      \item Raquette
    \end{itemize} \vspace{1em}

    Arduino
    \begin{itemize}
      \item Caméra
      \item Capteur de présence
    \end{itemize}
  \end{frame}

  \begin{frame}
    \frametitle{Modèle AADL}
    \framesubtitle{Implémentation logicielle}

    Système
    \begin{itemize}
      \item ISR1 I2C
      \item ISR1 Bluetooth
    \end{itemize} \vspace{1em}

    Applicatif
    \begin{itemize}
      \item Acquisition distance
      \item Acquisition consigne distance
      \item Asservissement roue
      \item Asservissement raquette
    \end{itemize}
  \end{frame}

  \begin{frame}
    \frametitle{Modèle AADL}
    \framesubtitle{Non-implémenté}

    Matériel
    \begin{itemize}
      \item Port USB
      \item Écran LCD
      \item Batterie
      \item Haut-parleur
    \end{itemize} \vspace{1em}

    Logiciel
    \begin{itemize}
      \item Tâche d’affichage
      \item Couche logicielle Arduino
    \end{itemize} \vspace{1em}

    Propriétés temporelles
    \begin{itemize}
      \item Traitements
      \item Communications
    \end{itemize}
  \end{frame}

  \begin{frame}
    \frametitle{Modèle AADL}
    \framesubtitle{Remarques et questions}

    Remarques
    \begin{itemize}
      \item AVR en tant que {\it processor} (pas {\it device})
      \item Choix de modélisation et description des propriétés
      %\begin{itemize}
      %  \item Pilotes en tant que {\it server subprograms}
      %  \item {\it bus accesses} dans les {\it feature groups}
      %  \item Arduino DIO en tant que {\it event data ports}
      %  \item Exclusivité sur les E/S du NXT
      %\end{itemize}
      \item Datasheet NXT incomplet
      \item Informations temporelles CMUcam
    \end{itemize} \vspace{1em}

    Questions
    \begin{itemize}
      \item Quel forme pour le démonstrateur ?
      \item Utilisation du tachymètre ?
      \item Quel niveau d’abstraction ?
    \end{itemize}
  \end{frame}

  %%%

  \begin{frame}
    \frametitle{Implémentation matérielle}
    \framesubtitle{Justification du format}

    Expérimentation de prototypes
    \begin{itemize}
      \item Perte de temps
    \end{itemize} \vspace{1em}

    Réalisation d’une réplique
  \end{frame}

  \begin{frame}
    \frametitle{Implémentation matérielle}
    \framesubtitle{Composition}

    Présent
    \begin{itemize}
      \item Roue
      \item Raquette
      \item Capteur de distance (ajout)
    \end{itemize} \vspace{1em}

    Absent
    \begin{itemize}
      \item Arduino
      \item Caméra
      \item Capteur de présence
      \item Capteur de distance doublé
      \item Capteurs de pression (retrait)
    \end{itemize}
  \end{frame}

  %%%

  \begin{frame}
    \frametitle{Implémentation logicielle}
    \framesubtitle{Réalisé}

    Tâches
    \begin{itemize}
      \item Affichage
      \item Initialisation
      \begin{itemize}
        \item Communication Bluetooth
        \item Communication I2C
        \item Tachymètre moteur
      \end{itemize}
      \item Acquisition distance
      \item Acquisition de la consigne de distance
      \item Asservissement de la roue
    \end{itemize}
  \end{frame}

  \begin{frame}
    \frametitle{Implémentation logicielle}
    \framesubtitle{Réalisé (suite)}

    Tâches (suite)
    \begin{itemize}
      \item Asservissement de la raquette
      \begin{itemize}
        \item Avec tachymètre
        \item Consigne {\it en dur}
        \item {\it Brouillon}
      \end{itemize}
    \end{itemize} \vspace{1em}

    Pilote Bluetooth
    \begin{itemize}
      \item Recherche origine problème
      \item Comparaison avec nxtOSEK
      \item Perte de temps
    \end{itemize}
  \end{frame}

  \begin{frame}
    \frametitle{Implémentation logicielle}
    \framesubtitle{Non-réalisé}

    \begin{itemize}
      \item Système de messagerie
      \item Acquisition double capteur
      \item Acquisition Arduino
      \vspace{1em}
      \item Pilote Bluetooth fonctionnel
      \item Pilote I2C fonctionnel
      \vspace{1em}
      \item Initialisation Bluetooth (boucle)
      \item Initialisation (StartUpHook)
    \end{itemize}
  \end{frame}

  %%%

  \begin{frame}
    \frametitle{Autres}
    \framesubtitle{}

    \begin{itemize}
      \item Enjeux du stage (génération)
      \item Réseaux de Petri
      \item Rapport préliminaire
      \begin{itemize}
        \item État de l'art
        \item Contribution(s) ?
      \end{itemize}
      \item (Pas de connexion filaire)
      %\item Exemples pour NXT pas compilable (dernière release)
      %\begin{itemize}
      %  \item Erreur génération Makefile
      %  \item Pas de compatibilité .oil et GOIL2
      %  \item Absence de typedef uintx et sintx
      %\end{itemize}
      %\item GOIL pas robuste
    \end{itemize}
  \end{frame}

\end{document}
